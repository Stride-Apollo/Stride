\documentclass{article}
\usepackage{multirow}

\title{Data format Stride checkpointing}
% \date{2017-03-09}
% \author{Federico Quin}


\begin{document}
	\maketitle
	\section{Hierarchy}
	\subsection{level /}
	\subsection{level /Timestep\_n}

	\section{Table structure}
		\begin{tabular}{ |l|l|l| }
		\hline

		\multirow{3}{*}{/configurations} & rank & 1 \\
		& dims & 1 \\
		& dtype & ConfDatatype \\ \hline

		\multirow{3}{*}{/amt\_persons} & rank & 1 \\
		& dims & 1 \\
		& dtype & H5T\_NATIVE\_UINT \\ \hline

		% TI = Time Independent
		\multirow{3}{*}{/personsTI} & rank & 1 \\
		& dims & amt\_persons \\
		& dtype & PersonTIDatatype \\ \hline

		\multirow{3}{*}{/amt\_timesteps} & rank & 1 \\
		& dims & n\_steps \\
		& dtype & H5T\_NATIVE\_UINT \\ \hline

		\multirow{3}{*}{/Timestep\_n/randomgen} & rank & 1 \\
		& dims & 1 \\
		& dtype & RNGDatatype \\ \hline

		% TD = Time Dependent
		\multirow{3}{*}{/Timestep\_n/PersonTD} & rank & 1 \\
		& dims & amt\_persons \\
		& dtype & PersonTDDatatype \\ \hline

		\multirow{3}{*}{/Timestep\_n/Calendar} & rank & 1 \\
		& dims & 1 \\
		& dtype & CalendarDatatype \\ \hline
		\end{tabular}
		\\ \\
		
		\subsection{Custom datatypes}
			\subsubsection{ConfDatatype}
				TBD
			\subsubsection{RNGDatatype}
				\begin{itemize}
					\item H5T\_NATIVE\_UINT - seed
					\item (things to hold state of generator)
				\end{itemize}
			\subsubsection{PersonTIDatatype}
				\begin{itemize}
					\item H5T\_NATIVE\_UINT - ID
					\item H5T\_NATIVE\_DOUBLE - age
					\item H5T\_NATIVE\_CHAR - gender

					\item H5T\_NATIVE\_UINT - householdID
					\item H5T\_NATIVE\_UINT - schoolID
					\item H5T\_NATIVE\_UINT - workID
					\item H5T\_NATIVE\_UINT - primcommID
					\item H5T\_NATIVE\_UINT - seccommID
				\end{itemize}
			\subsubsection{PersonTDDatatype}
				\begin{itemize}
					\item H5T\_NATIVE\_HBOOL - athousehold
					\item H5T\_NATIVE\_HBOOL - atschool
					\item H5T\_NATIVE\_HBOOL - atwork
					\item H5T\_NATIVE\_HBOOL - atprimcomm
					\item H5T\_NATIVE\_HBOOL - atseccomm
					\item H5T\_NATIVE\_HBOOL - participant
				\end{itemize}
			\subsubsection{CalendarDatatype}
				\begin{itemize}
					\item H5T\_NATIVE\_HSIZE - day
					\item Compound Datatype := dateString (8x H5T\_NATIVE\_CHAR) - date
							\\ \textit{Date in iso format, string of 8 characters.}

				\end{itemize}
\end{document}